\documentclass[a4paper,parskip=full,oneside]{scrartcl}

\usepackage[utf8]{inputenc}
\usepackage[T1]{fontenc}

\usepackage{lmodern}
\usepackage{mathpazo}
\usepackage{microtype}

\usepackage[hidelinks]{hyperref}
\usepackage{textcomp}

\usepackage[naustrian]{babel}
%\usepackage{listings}

\subject{
	Projektkooperation der FH-Bachelorstudiengänge \\
	Militärische Führung, Wr. Neustadt \\
	Software Engineering, FH OÖ, Campus Hagenberg}
\title{Schnittstellenspezifikation Sensorik}
\subtitle{Datenaustausch Schuhfried Biofeedback Hardware}
\date{\vspace{-2\baselineskip}}
%\author{Josef Frauscher, Sandra Horner, Christian Neumüller}

\addtokomafont{subject}{\small}

\begin{document}
\maketitle

\begin{center}
\begin{tabular}{ll}
Version: & 1.2 \\
Ersteller: & Christian Neumüller \\
Erstelldatum: & 10. April 2015 \\
Letzte Änderung: & \today \\
\end{tabular}
\end{center}
\bigskip
\tableofcontents

\pagebreak
\section{Einleitung}
Für die Projektkooperation zwischen dem FH-Bachelorstudiengang Software
Engineering der Fachhochschule OÖ, Campus Hagenberg (uns) und dem
FH-Bachelorstudiengang Militärische Führung, Wr. Neustadt auf dem Gebiet
des Führungskräftetrainings mittels Virtual Reality und Sensorauswertung zur
quantitativen und qualitativen Bestimmung der emotionalen Betroffenheit werden
unterschiedliche Sensorsysteme eingesetzt.

Wie wir von Oberst Dr. Slanic erfahren haben, hat der FH-Bachelorstudiengang
Militärische Führung, Wr. Neustadt zwei Biofeedback 2000\textsuperscript{x-pert} Systeme
angekauft und im Einsatz befindlich.

Der Studiengang SE in Hagenberg verfügt über modulare g.tech Sensoren, die auch während
der Projektentwicklung eingesetzt werden.

Um die zu entwickelnde Auswertesoftware mit beiden Systemen kompatibel zu gestalten,
benötigen wir eine Definition der Schnittstellen, über die wir die Biofeedback Sensordaten
abrufen können.

Auf Anweisung von Dr. Slanic übermitteln wir mit diesem Dokument eine Schnittstellenspezifikation
und treten direkt an Sie heran.

Wir haben für diesen Zweck ein einfaches binäres Datenformat entwickelt,
das nachfolgend spezifiziert wird.

\section{Übertragung}

Für den produktiven Einsatz ist eine asynchrone Übertragung der Sensorpakete vorgesehen.
Das derzeit angestrebte Übertragungsprotokoll wäre dabei Bluetooth oder Ethernet
(Realisierung eventuell von der Biofeedback-Analysestation hin zum spezifizierten Host über UDP)
Die Sensordaten werden dabei in einem festen Takt gestreamt.

Für die Entwicklung wäre es derzeit hilfreich, eine Datei mit aufgezeichneten Sensordaten
zur Verfügung gestellt zu bekommen (nur die unten spezifizierten Nutzdaten).

Mit dem Recording kann einerseits die Weiterverarbeitung der Sensordaten realisiert,
und andererseits ein Simulator für die Bluetooth-Übertragung entwickelt werden.

\section{Steuerbefehl}

Die Notwendigkeit eines einzigen Befehls zum Starten der Übertragung besteht.
Konfiguration, Sensorverwaltung, etc. wird weiterhin über die Biofeedback-Applikation durchgeführt.

Eine Realisierung lässt sich auf unterschiedliche Arten bewerkstelligen:
\begin{itemize}
\item Steuerbefehl über Netzwerkprotokoll übertragen (\texttt{start <Host-IP> <Host-Port>})
\item Konfigurationserweiterung der Biofeedback-Applikation oder zusätzliches Control
\end{itemize}


\section{Dateistruktur}

Eine Binärdatei muss nach folgender Struktur (in EBNF-Notation) aufgebaut sein:

\begin{verbatim}
Datei := Header {Sample}

Header := Formatversion Samplerate Startzeitpunkt
Formatversion := "phr1"
Samplerate := uint32
Startzeitpunkt := uint64

Sample := Samplenummer Puls Hautleitwert Atemfrequenz Temperatur
Samplenummer := uint64
Puls := double
Hautleitwert := double
Atemfrequenz := double
Temperatur := double
\end{verbatim}

Die Übertragung der UDP-Pakete erfolgt analog zum oben skizzierten Binärformat.
Einzige Änderung: Es wird immer nur ein Nutzdatensample pro Paket übertragen.

\begin{verbatim}
Paket := Header Sample
\end{verbatim}

Die Übertragung des Headers bei jedem Paket ist als massiv redundant anzusehen,
erlaubt jedoch die Behandlung etwaiger Fehler.

\section{Wertformat}

Alle Daten sind in Big Endian zu kodieren.

Die Formatversion versteht sich als Präambel statischer Länge
(ohne z.B. \textbackslash{}0 am Ende).

Die Samplerate ist in Hertz (Samples pro Sekunde) angegeben.

Die Samplenummer ist eine fortlaufend vergebene Nummer die bei $0$ beginnt,
so dass dem Sample mit der Nummer $i$ der Zeitpunkt
$\mathit{Startzeitpunkt\ in\ Sekunden} + i \cdot \frac{1}{\mathit{Samplerate}}$
zugeordnet werden kann.

\texttt{double} ist eine nach IEC 60559:1989 (IEEE 754) kodierte Gleitkommazahl \cite{ieee745}.

Der \texttt{Startzeitpunkt} ist eine Windows \texttt{FILETIME},
also die Anzahl der 100-Nanosekunden
Intervalle, die seit 12:00 Mitternacht am 1. Jänner 1601
Coordinated Universal Time (UTC) vergangen sind \cite{filetime}.

\subsection{Einheiten der Messgrößen}
Die Nutzdaten entsprechen folgenden Einheiten:

\begin{tabular}{llc}
Puls & Hertz & Hz \\
Atemfrequenz & Hertz & Hz \\
Temperatur & Grad Celsius & °C \\
Hautleitwert & Siemens & S \\
\end{tabular}

\begin{thebibliography}{9}

\bibitem{ieee745}
  IEEE Computer Society,
  \emph{IEEE Standard for Binary Floating-Point Arithmetic},
  The Institute of Electrical and Electronics Engineers, Inc 345 East 47th Street, New York, NY 10017, USA,
  1985, \url{http://ieeexplore.ieee.org/stamp/stamp.jsp?tp=&arnumber=30711}, zuletzt abgerufen am 17. April 2015.

\bibitem{filetime}
	Microsoft
	\emph{File Times}
	\url{https://msdn.microsoft.com/en-us/library/windows/desktop/ms724290.aspx}
\end{thebibliography}


\end{document}